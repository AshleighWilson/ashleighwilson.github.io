\documentclass{minimal}
\usepackage[RPvoltages]{circuitikz}
\thispagestyle{empty}

\newcommand\bjtname[1]{($(#1.C)!0.5!(#1.E)$) node[anchor=west]{} }

\begin{document}
    \begin{circuitikz}[american, cute inductors]
        \node [op amp](A1){\texttt{OA1}};
        \draw (A1.-) to[short] ++(0,1) coordinate(tmp) to[R, l_=$R$] (tmp -| A1.out) to[short] (A1.out);
        \draw (tmp) to[short] ++(0,1) coordinate(tmp) to[C=$C$] (tmp -| A1.out) to[short] (A1.out);
        \draw (A1.+) to [battery2, invert] ++(0,-2.5) node[ground](GND){};
        \draw (A1.-) to [L=$L$] ++(-2,0) coordinate(tmp) to[sV, l=$v_s$, fill=yellow] (tmp |-GND) node[ground]{};
        \draw (A1.out) to[R=$R_s$] ++(2,0) coordinate(bb) to[I, l_=$I_B$, invert] ++(0,2) node[vcc](VCC){};
        \draw (bb) to[D, l=$D$, *-] ++(0,-2) coordinate(bb1) to[R=$R_m$] ++(0,-2) node[vee](VEE){};
        \draw (bb) --++(1,0) node[npn, anchor=B](Q1){} \bjtname{Q1};
        \draw (bb1) --++(1,0) node[pnp, anchor=B](Q2){} \bjtname{Q2};
        \draw (Q1.E) -- (Q2.E) ($(Q1.E)!0.5!(Q2.E)$) to [short, *-o, name=S] ++(2.5,0)
        node[right]{$v_{o_Q}$};
        \draw (S.s) to[european resistor, l=$Z_L$, *-] (S.s|-GND) node[ground]{};
    \end{circuitikz}
\end{document}
